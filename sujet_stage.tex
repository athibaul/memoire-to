\documentclass{article}

\usepackage[utf8]{inputenc}
\usepackage[francais]{babel}

\title{Algorithmes proximaux à métrique on euclidienne et application au transport optimal}
\date{}

\begin{document}

\maketitle

L'objectif du stage est d'étudier les méthodes d'optimisation proximales pour des problèmes de minimisation dans le cadre convexe et non lisse.\\
L'utilisation de métriques non euclidiennes peut rendre le calcul de l'opérateur proximal de certaines fonctions explicite [1,2], ce qui permet d'obtenir des algorithmes de minimisation plus rapides en pratique.\\
Il sera intéressant de regarder l'intégration de métriques non euclidiennes (comme l'entropie négative) pour différents types d'algorithmes proximaux [3] et de faire le lien avec l'algorithme de Sinkhorn,
utilisé notamment pour le calcul de coût de transport optimal [4].

\textbf{Dates~:} 5 juin-28 juillet environ.

\textbf{Encadrants~:} Charles Dossal (Maitre de Conférence, Université de Bordeaux), Nicolas Papadakis (Chargé de recherche CNRS).

\textbf{Lieu~:} Institut de Mathématiques de Bordeaux, UMR 5251, 351 Cours de la Libération, 33405 Talence Cedex

\begin{thebibliography}{9}
\bibitem{b1}
A. Chambolle, T. Pock.  On the ergodic convergence rates of a first-order primal-dual algorithm.  Journal of Mathematical Programming, 2016.
\bibitem{b2}
H. Baushcke, J. Bolte, M. Teboulle. A Descent Lemma Beyond Lipschitz Gradient Continuity: First-Order Methods Revisited and Application. MATHEMATICS OF OPERATIONS RESEARCH, 2016
\bibitem{b3}
H. Raguet, J. Fadili, G. Peyré. A Generalized Forward-Backward splitting, SIAM Journal on Imaging Sciences, 2013.
\bibitem{b4}
J-D. Benamou, G. Carlier, M. Cuturi, L. Nenna, G. Peyré. Iterative Bregman Projections for Regularized Transportation Problems. SIAM Journal on Scientific Computing, 2015.
\end{thebibliography}

\end{document}