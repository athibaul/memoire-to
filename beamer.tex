% !TEX TS-program = pdflatex
% !TEX encoding = UTF-8 Unicode



\documentclass{beamer}



\mode<presentation>
{
  \usetheme{Szeged}
}


\usepackage[utf8]{inputenc}
\usepackage[T1]{fontenc}
\usepackage[francais]{babel}
\usepackage[]{amsmath}
\usepackage[]{amsfonts}
\usepackage{amssymb}
\usepackage{amsthm}
\usepackage{enumerate}
\usepackage{graphicx}
\usepackage{bbm}
\usepackage[margin=1cm]{caption}


% Pour utiliser le symbole "non équivalent"
\DeclareMathOperator{\nequiv}{
    \Longleftrightarrow\kern-1.3em \not \kern1.3em
}
\DeclareMathOperator*{\argmin}{argmin}
\newcommand{\transpose}{\intercal}
\DeclareMathOperator{\IR}{\mathbb{R}}
\DeclareMathOperator{\IN}{\mathbb{N}}
\DeclareMathOperator{\One}{\mathbbm{1}}

\DeclareMathOperator{\diag}{\mathbf{diag}}
\DeclareMathOperator{\Ccal}{\mathcal{C}}





\title{Transport optimal et régularisation}

\subtitle{L3 Maths-Info}

\author{Alexis \textsc{Thibault}}

\institute{ENS}

\date{Mai 2017}

\AtBeginSection[]
{
  \begin{frame}<beamer>{Plan}
    \tableofcontents[currentsection]
  \end{frame}
}


% If you wish to uncover everything in a step-wise fashion, uncomment
% the following command: 

%\beamerdefaultoverlayspecification{<+->}


\begin{document}

\begin{frame}
  \titlepage
\end{frame}

\begin{frame}{Plan}
  \tableofcontents
\end{frame}


% Structuring a talk is a difficult task and the following structure
% may not be suitable. Here are some rules that apply for this
% solution: 

% - Exactly two or three sections (other than the summary).
% - At *most* three subsections per section.
% - Talk about 30s to 2min per frame. So there should be between about
%   15 and 30 frames, all told.

% - A conference audience is likely to know very little of what you
%   are going to talk about. So *simplify*!
% - In a 20min talk, getting the main ideas across is hard
%   enough. Leave out details, even if it means being less precise than
%   you think necessary.
% - If you omit details that are vital to the proof/implementation,
%   just say so once. Everybody will be happy with that.

\section{Transport optimal}

\subsection{Problème de Monge}

\begin{frame}{Question initiale}

	Répartition de masse = histogramme = mesure de probabilité.
	
    Problème : comment transporter une mesure de probabilité $\mu$ vers une autre $\nu$ efficacement ?
	\pause
    \begin{itemize}
    \item Monge : par une application 
    $ T :     \IR^d     \rightarrow     \IR^d $ \og proche \fg de l'identité
    \end{itemize}
    \begin{center}\includegraphics[height=4cm]{transport_monge2.png}\end{center}
    
\end{frame}

\begin{frame}{Question initiale}

	Répartition de masse = histogramme = mesure de probabilité.
	
    Problème : comment transporter une mesure de probabilité $\mu$ vers une autre $\nu$ ?
    \begin{itemize}
    \item Kantorovich : par un plan de transport
    \includegraphics[height=5cm]{1d_interp_2.png}
    \end{itemize}
    
\end{frame}

\subsection{Problème de Kantorovich}

\begin{frame}{Formulation de Kantorovich}
\[
\begin{split}
\Pi(\mu,\nu) = \{\gamma\;\text{mesure de probabilité sur}\,X\times Y\;| \\
\;(\pi_X)_\# \gamma = \mu,\; (\pi_Y)_\# \gamma = \nu\}
\end{split}
\]
	\begin{itemize}
	\item Symétrique
	\item Permet de séparer les masses de Dirac
	\item Adapté au cas discret
	\end{itemize}
\end{frame}

\begin{frame}{Question de minimisation}
\begin{itemize}
\item $c$ coût de transport ponctuel (exemple : distance)
\item Coût d'un plan de transport~:
\[
K(\gamma) = \int_{X\times Y} c\,d\gamma \; ,\quad \text{où} \quad \gamma \in \Pi(\mu,\nu)
\]
\pause
\item Coût de transport optimal~:
\[C(\mu,\nu) = \min_{\gamma \in \Pi(\mu,\nu)} K(\gamma) \]
\pause
\item Distance (symétrique, définie, inégalité triangulaire)
\end{itemize}
\end{frame}

\begin{frame}{Calcul numérique}
\begin{itemize}
	\item Problème de programmation linéaire
	\item Méthode du simplexe $\rightarrow$ compliqué et lent pour $n>100$
	\item Approximation ?
\end{itemize}
\end{frame}



\section{Problème régularisé}

\subsection{Transport optimal régularisé}

\begin{frame}{Pénalisation par l'entropie}
\begin{itemize}
\item Précédemment~:
\[K(\gamma) = \sum_{i,j} c_{i,j} \gamma_{i,j} \]
\pause
\item Maintenant~:
\begin{equation*}\label{eq:keps}
\begin{split}
K^\epsilon (\gamma) 
 &= K(\gamma) + \epsilon S(\gamma) \\
&= \sum_{i,j} c_{i,j} \gamma_{i,j} + \epsilon \sum_{i,j} \gamma_{i,j} \log(\gamma_{i,j}) 
\end{split}\end{equation*}
\item $\gamma^\epsilon := \argmin K^\epsilon $
\end{itemize}
\end{frame}

\begin{frame}{Conséquences}
\begin{itemize}
\item Coefficients tous non nuls
\item Lissage du plan optimal
\begin{tabular}{c c}
\includegraphics[width=4.5cm]{1d_interp_0_5.png} & 
\includegraphics[width=4.5cm]{1d_interp_18.png} \\
$\epsilon = 2$ & $\epsilon = 0.05$
\end{tabular}
\item Convergence vers le \og vrai \fg plan optimal pour $\epsilon \rightarrow 0$
\end{itemize}
\end{frame}



\subsection{Algorithme de Sinkhorn}

\begin{frame}{Forme de la solution}
\begin{itemize}
\item Coefficients non nuls $\rightarrow n^2$ degrés de liberté ?
\item En fait,
\[\exists a,b, \quad \gamma^\epsilon = \diag(a) G \diag(b) \]
où $G_{i,j} = \exp(-c_{i,j}/\epsilon)$
\item Seulement $2n$ degrés de liberté
\end{itemize}
\end{frame}

\begin{frame}{Algorithme}
\begin{table}[h!]
\centering
\caption*{Algorithme de Sinkhorn}
\label{algorithme}
\begin{tabular}{|l|}
\hline
\'Etant donnés ($(c_{i,j})$, $\epsilon$, $(\mu_i)_{1\le i \le n}$,$(\nu_j)_{1 \le j \le m}$), on pose~: \\
$a^0 = (1/n, \ldots, 1/n)$\\
$(G_{i,j}) = (e^{c_{i,j}})$\\
Puis on itère (jusqu'à ce qu'un critère d'arrêt convenable soit vérifié)~: \\
$\quad b^{k+1}_j = \nu_j / (G^\transpose a^k)_j$\\
$\quad a^{k+1}_i = \mu_i /(G b^{k+1})_i$ \\
\hline
\end{tabular}
\end{table}
\end{frame}

\begin{frame}{Avantages de l'algorithme}
\begin{itemize}
\item Simple ($\ne$ prog. linéaire)
\item Calculs matriciels $\rightarrow$ GPU
\item Généralisations faciles : algorithme similaire pour calculer des barycentres de transport optimal
\end{itemize}
\end{frame}

\section{Applications}

\subsection{Interpolation de formes}

\begin{frame}{Démonstration de l'interpolation de formes}
\texttt{barycentre\_demo.py}
\end{frame}

\subsection{Nuages de mots}

\begin{frame}{Word Embedding}
Word Embedding = plongement dans un espace de dimension $N = 300$
\begin{itemize}
\item Synonymes proches (ex : frog $\rightarrow$ frogs, toad, litoria, leptodactylidae, rana, lizard, eleutherodactylus)
\item Relations similaires $\rightarrow$ vecteurs similaires
\end{itemize}
\begin{figure}[h!]
\centering
\fbox{\includegraphics[width=3cm]{man_woman.jpg}}
\hspace{1cm}
\fbox{\includegraphics[width=3cm]{comparative_superlative_small.jpg}}
\end{figure}
\end{frame}

\begin{frame}{Idée}
\begin{itemize}
\item Texte $\rightarrow$ fréquences
\item Plongement lexical $\rightarrow$ distance $\rightarrow$ coût de transport
\item Distance sémantique entre textes
\item Interpolation sémantique de textes
\end{itemize}
\end{frame}

\begin{frame}{Démonstration}
\texttt{glove\_demo.py}
\end{frame}

\begin{frame}{Observations}
\begin{itemize}
\item Mots \og neutres \fg privilégiés
\pause
\item Distances entre textes
	\begin{itemize}
	\item Uniquement basé sur le champ lexical
	\item Classification par auteur
	\pause
	\item Rapprochements intéressants~: Woolf, Stevenson et J.K. Rowling
	\end{itemize}
\end{itemize}
\end{frame}


\section*{Conclusion}

\begin{frame}{Conclusion}

  \begin{itemize}
  \item
    Transport optimal $\rightarrow$ \alert{distance} entre mesures de probabilité.
  \item
	T.O. régularisé $\rightarrow$ distances et barycentres \alert{faciles à calculer} par Sinkhorn.
  \item
    \alert{Outil puissant} pour des tâches variées~: traitement d'images, machine learning, économie, astrophysique...
  \end{itemize}
  
\end{frame}



% All of the following is optional and typically not needed. 
\appendix
\section<presentation>*{\appendixname}
\subsection<presentation>*{Fin}

\begin{frame}{Merci !}
\begin{center}
\includegraphics[height=6cm]{ENS.png}
\end{center}
\end{frame}

\end{document}


